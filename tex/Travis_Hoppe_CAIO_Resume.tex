\documentclass[]{scrartcl}
\usepackage{src/cleancv}
\usepackage{enumitem}

\newcommand{\CVtype}{Federal Resume}
\newcommand{\CVname}{Travis Aaron Hoppe}
\newcommand{\CVemail}{travis.hoppe@gmail.com}
\newcommand{\CVdegrees}{PhD Physics}
\newcommand{\CVphone}{$(775) \ 287\hbox{-}4033$}

\titleformat{\section}[block]{\noindent\bfseries}{}{0em}{} 
\titleformat{\subsection}[block]{\noindent\itshape}{}{0em}{} 

\titlespacing*{\section}
{0pt}{2ex}{1ex}
\titlespacing*{\subsection}
{0pt}{1ex}{1ex}

\begin{document}
\begin{cleanCV}

   \SectionHead{\MakeUppercase{Highlights}}
   \vspace{-1em}

   {
  \vspace{-0.25em}
  \begin{itemize}

  \item Federal expert in artificial intelligence (AI), machine learning (ML), and data science. Led multiple teams across CDC to develop and implement cutting-edge policies, guidance, and strategies for AI/ML adoption, including conversational AI (Chat GPT), and infrastructure needs to support AI/ML implementation. Regular presenter across federal and industry venues on topics around Trustworthy AI, implementation, and development. Extensive experience in presenting to federal and industry audiences around Trustworthy AI, implementation, and development of new methodology.

   \item Respected and well cited author in AI/ML and biomedical research to address questions on multiple topics including: disparities in NIH funding (\href{https://www.science.org/doi/10.1126/sciadv.aaw7238}{Topic Choice}: 400 citations, key evidence used by the \href{https://bluntrochester.house.gov/uploadedfiles/191220_ltr_to_nih_about_grant_disparities.pdf}{Congressional Black Caucus}), open citations (\href{https://www.ncbi.nlm.nih.gov/pmc/articles/PMC6786512/}{NIH Open Citation Collection}: 70 citations), and a previously state-of-the-art AI dataset (\href{https://arxiv.org/abs/2101.00027}{The Pile}: 390 citations).

  \end{itemize}
}

   \SectionHead{
     \MakeUppercase{Professional Experience} \newline
     \MakeUppercase{ECQ}s~\&~\MakeUppercase{MTQ}s highlighted
   }
  \vspace{-1em}

\WorkExperience
{}
{Assistant Director of AI Research and Development}
{
  \newline White House Office of Science and Technology Policy (OSTP)
  \newline February 2023 - Present (detailed from CDC)
  \newline ST 1560
}
{
  \vspace{-0.25em}
  \begin{itemize}

  \item \textbf{Established organizational vision} for the nation's AI R\&D as a co-chair for the National Science and Technology Council (NSTC) subcommittee AI and ML. Wrote and negotiated the adoption of the 2024-2027 work plan establishing a vision for data, code, and AI models, stronger international engagements, and improved Federal AI reporting across research, development, and application.
  \item \textbf{Built coalitions} across the Federal government by connecting State, Tribal, Local, and Territorial governments through many engagements via the White House \href{https://www.whitehouse.gov/ostp/news-updates/2023/11/13/readout-of-roundtable-with-state-local-tribal-and-territorial-leaders-on-tech-policy/}{Tech Policy Network}, including the National League of Cities, the American Public Human Services Association, and the Council of State and Territorial Epidemiologists
  \item \textbf{Drove results} for major White House initiatives including the National AI Research Resource (NAIRR), the National Secure Data Service (NSDS), and the OMB Memorandum (\href{https://www.whitehouse.gov/wp-content/uploads/2024/03/M-24-10-Advancing-Governance-Innovation-and-Risk-Management-for-Agency-Use-of-Artificial-Intelligence.pdf}{M-24-10}), ``Advancing Governance, Innovation, and Risk Management for Agency Use of Artificial Intelligence'', by aligning stakeholders across the various regulations, legislation, and international documents.
  \item \textbf{Managed, prioritized, and directed resources} for projects within agencies including the Federal AI use-case inventory, the Networking and Information Technology Research and Development (NITRD) AI program repository, the Chief AI Officer's Council working groups, and \href{https://www.federalregister.gov/documents/2023/11/01/2023-24283/safe-secure-and-trustworthy-development-and-use-of-artificial-intelligence}{Executive Order 14110}, `` Presidential Document Safe, Secure, and Trustworthy Development and Use of Artificial Intelligence''.

  \end{itemize}
}

\WorkExperience
{}
{Associate Director for Data Analytics and Data Science (ADDADS)}
{
  \newline Centers for Disease Control and Prevention (CDC), National Center for Health Statistics (NCHS)
  \newline December 2023 - Present (on detail to OSTP)
  \newline ST 1560
  \newline
  \newline Senior Service Fellow
  \newline May 2022 - December 2023
  \newline RG GS-15/2 1530
}
{
  \vspace{-0.25em}
  \begin{itemize}

  \item \textbf{Led organizational change} across CDC to achieve the distinction of being the first Federal agency to fully adopt generative AI to all staff. Accomplished results by leading 15 pilots programs across the Centers, coordinating financial, human, and IT resources, cleaning the program through cybersecurity, legal, and labor concerns.

  \item \textbf{Modernized} NCHS data systems through cloud adoption through a multi-year plan. Aligned and coordinated the plan with CDC's data modernization initiative (DMI) to help build the foundation for data sharing across all levels of public health. Coordinated policy, governance, and legal issues around the usage of CIPSEA protected data. Worked with Office of Management and Budget (OMB) to adopt CIPSEA guidance and lead NCHS as an early cloud adopter for statistical agencies.
    
  \item \textbf{Piloted} innovative AI, ML, and data science projects to help identity, analyze, and report on emerging public health data. Projects include an item non-response detection model for survey text which resulted in the \href{https://www.cdc.gov/nchs/data-science/SANDS-model-context.htm}{first AI model} released by NCHS, privacy enhancing technologies (PETs) like homomorphic encryption and privacy preserving record linkage, text-to-speech transcription which resulted in a 10-fold improvement from prior methods, and the creation of a bibliometrics dashboard for reporting.

      \item \textbf{Built community} by starting and leading three different inclusive organizations within the agency: NCHS Innovation, NCHS Data Science Community of Practice, and the agency-wide EDAV Best Practices group. Organizations helped build community, foster communication, and spur innovative practices across CDC.
    
  \item \textbf{Implemented} the first set of model standards for AI/ML within the agency for trustworthy, responsible, and ethical usage. Aligned with existing standards including the NIST AI RMF and HHS Trustworthy AI and the organizational of CDC and NCHS. 

  \item \textbf{Harmonized} metadata usage for all NCHS datasets published on \href{https://data.cdc.gov/browse?category=NCHS}{data.cdc.gov}. Developed and implemented standards for tagging datasets around Social Determinants of Health (SDoH).
    
      \item \textbf{Served} as an agency representative for National Science and Technology Council (NSTC) subcommittee on AI/ML and regular presenter at HHS AI and GSA AI communities of practice. Active committee member of the Federal Committee on Statistical Methodology (FCSM) co-leading a Federal level metadata initiative.

  \end{itemize}
}

\vspace{1em}

    
  \WorkExperience
{}
{Senior Service Fellow: Chief Data Scientist}
{
  \newline Centers for Disease Control and Prevention (CDC), National Center for
Health Statistics (NCHS)
  \newline September 2020 - May 2022
  \newline Title 42 (GS-14/5 equivalent)
}
{
  \vspace{-0.25em}
  \begin{itemize}

   \item \textbf{Lead} for NCHS Data Modernization Initiative: \emph{Increase Use, Discoverability, and Access to NCHS Data}. Established pilot projects and worked with stakeholders to create statements of work and business needs.
   \item \textbf{Advised} on CDC's response to the Executive Order on AI and the NCHS Data Science Strategic Plan. Presented to Board of Scientific Council, NCHS All-Hands, and delivered subject matter talks on Bias in AI, advances in Natural Language Processing, bibliometrics, and more.
   \item \textbf{Developed and coordinated} Center-wide metadata standards and built a custom ontology using evidence based sources: publications, web searches, and market research.
   \item \textbf{Researched} new methodology to study free text responses from the Research and Development Survey (RANDS), including non-response detection and zero-shot learning objectives.
   \item \textbf{Implemented} PII detection processes for restricted microlevel data to allow privacy preserving research and development.
  \end{itemize}
}
  
  \WorkExperience
{}
{Senior Data Scientist / Portfolio Analyst}
{
  \newline National Institutes of Health (NIH) / Division of Program Coordination, Planning, and Strategic Initiatives (DPCPSI) / Office of Portfolio Analysis (OPA) contracted under Lexical Intelligence
  \newline February 2016 - February 2020
}
{
  \vspace{-0.25em}
  \begin{itemize}
    
    \item Scientific team leader for a novel inter-agency government blockchain to detect grant duplication with minimal shared data. Coordinated research, oversaw design, and developed protocols within the NIH and National Science Foundation (NSF) teams.

    \item Developed new analytic tools to process the text of NIH grants and publications using distributional embeddings (word2vec) and transformers (BERT). Tools were deployed for analysis presented to NIH senior leadership, Congress, and publications in high-ranking journals.

    \item Architected and productionized machine learning models for classification, regression, outlier detection, and language modeling. Creator and maintainer of several open-source tools used internationally in the scientific community.

      \item Trained and mentored junior staff in natural language processing (NLP) and machine learning.
  
  \item Analyzed grant and publication portfolios, evaluating metrics such as clinical impact, technological impact, and award rates to build quantitative comparisons between various populations.
    
  \item Restored historical texts from books and generated new structured data from free text. Expanded NIH grant coverage by thirty years from archival documents. Cross-linked publications to an NIH application's biographical sketch and literature cited. Data used internally with the NIH for analysis on racial disparity, topic analysis, mentorship, and grant efficacy.    
    
  \end{itemize}
}

\vspace{0.5em}
 
\WorkExperience
{}
{Postdoctoral Fellowship (IRTA) at National Institutes of Health}
{\newline Research Scientist}
{
  \newline April 2014 - February 2016
}
{
  \begin{itemize}
  \item Researched novel integration schemes for molecular dynamics simulations (MDS). Developed protein models for tertiary structure prediction from primary sequence.

  \item Designed and managed high-performance computing models on the NIH supercomputer, Biowulf. First to investigate containerized solutions for MDS using a graphics processing card.

  \item Worked in collaboration with experimentalists to test and validate models. 

  \end{itemize}
}

%\newpage

\WorkExperience
{}
{Postdoctoral Fellowship at National Institutes of Health}
{\newline Research Scientist}
{
  \newline August 2011 - April 2014
}
{
  \begin{itemize}
  \item Developed multi-scale theoretical and computational models to study protein folding, structure, and protein-protein. Derived hard-sphere models to account for crowding in biomolecular simulations and potentials to model anisotropic charge distributions.
  \item Managed large-scale parallel projects (1000+ cores) to simulate the cellular environment.
  \end{itemize}
}

%%%%%%%%%%%%%%%%%%%%%%%%%%%%%%%%%%%%%%%%%%%%%%%%%%%%%%%%%%%%%%%%%%%%%%%%%%%%%%%

\SectionHead{\MakeUppercase{Education}}

\WorkExperience
{2011}
{Doctor of Philosophy, Physics}
{
\\Drexel University
\\\emph{On the Role of Entropy in the Protein Folding Process}, \href{https://idea.library.drexel.edu/islandora/object/idea:3488}{Thesis}
}

\WorkExperience
{2008}
{Master of Science, Physics}
{\\Drexel University}

\WorkExperience
{2005}
{Bachelor of Science, Physics}
{\\University of Nevada}

\WorkExperience
{2005}
{Bachelor of Science, Mathematics}
{\\University of Nevada}
%%%%%%%%%%%%%%%%%%%%%%%%%%%%%%%%%%%%%%%%%%%%%%%%%%%%%%%%%%%%%%%%%%%%%%%%%%%%%%%

\SectionHead{\MakeUppercase{Committees served}}

\WorkExperienceX
{2021-Present}
{National Science and Technology Council (NSTC) subcommittee on AI/ML}
{Co-chair (2024-present), CDC representative (2021-2024)}

\WorkExperienceX
{2021-Present}
{Federal Committee on Statistical Methodology (FCSM)}
{Board Member}

\WorkExperienceX
{2022-2024}
{Building Trust and FAIRness into the Process for Finding and Using Government Data (Chief Data Officers Council and FCSM)}
{Co-lead}

\WorkExperienceX
{2023}
{Utility and Risks to CDC of Conversational Artificial Intelligence (AI) Technologies like Chat GPT (CDC)}
{Tiger Team Lead}

\WorkExperienceX
{2023}
{Supporting the U.S. Public Health Workforce, President's Council of Advisors on Science and Technology (PCAST)}
{External Expert}



\WorkExperienceX
{2022}
{Analytics and Machine Learning Implementation within CDC's Cloud Environment (CDC)}
{Tiger Team Lead}

\WorkExperienceX
{2021}
{National Science and Technology Council (NSTC): Epidemic Modeling and Forecasting Fast Track Action Committee (FTAC)}
{Contributor: Plan to Advance Data Innovation}

\WorkExperienceX
{2021}
{Health and Human Services (HHS): Open Data Task Force}
{Committee member}

\WorkExperienceX
{2021}
{CDC: Information Technology and Data Governance (ITDG)}
{Committee member}


%%%%%%%%%%%%%%%%%%%%%%%%%%%%%%%%%%%%%%%%%%%%%%%%%%%%%%%%%%%%%%%%%%%%%%%%%%%%%%%
\SectionHead{\MakeUppercase{Selected \mbox{Publications}}}

\vspace{0.5em}
\Paper
{2024}
{Semi-Automated Nonresponse Detection for Open-text Survey Data}
{Kristen Cibelli Hibben, Zachary Smith, Ben Rogers, Valerie Ryan, Paul Scanlon, Travis Hoppe}
{https://doi.org/10.1177/08944393241249720}
{Social Science Computer Review}

\Paper
{2023}
{Predicting causal citations without full text}
{Travis Hoppe, Salsabil Arabi, Ian Hutchins}
{https://www.pnas.org/doi/10.1073/pnas.2213697120}
{Proceedings of the National Academy of Sciences of the United States of America}

\Paper
{2023}
{A Framework for Data Quality: Case Studies}
{Lisa Mirel, Darius Singpurwalla, Travis Hoppe, Rolf Schmitt, Julie Weber, Erika Liliedahl}
{https://doi.org/10.21949/1529869}
{Federal Committee on Statistical Methodology}
%{Data Quality Framework Implementation Subcommittee}

\Paper
{2023}
{Dark citations to Federal resources and their contribution public health}
{Jessica Keralis, Juan Albertorio-Díaz, \& Travis Hoppe}
{https://doi.org/10.3389/frma.2023.1235208}
{Frontiers in Research Metrics and Analytics}
%{https://www.biorxiv.org/content/10.1101/2023.03.26.533809v1}{bioRxiv}

\Paper
{2019}
{Topic Choice Contributes to Lower Rate of NIH Awards to African-American/Black Scientists}
{Travis Hoppe, Aviva Litovitz, Kristine Willis, Rebecca Meseroll, Matthew Perkins, B. Ian Hutchins, Alison Davis, Michael Lauer, Hannah Valantine, James Anderson, \& George Santangelo}
{https://advances.sciencemag.org/content/5/10/eaaw7238}
{Science Advances}

\Paper
{2020}
{The Pile: An 800GB Dataset of Diverse Text for Language Modeling}
{Leo Gao, Stella Biderman, Travis Hoppe, \etal}
{https://arxiv.org/abs/2101.00027}
{arXiv}

\Paper
{2019}
{The NIH Open Citation Collection: A public access, broad coverage resource}
{Ian Hutchins, Kirk Baker, Matthew Davis, Mario Diwersy, Ehsanul Haque, Robert Harriman, Travis Hoppe, Stephen Leicht, Payam Meyer, George Santangelo}
{https://journals.plos.org/plosbiology/article?id=10.1371/journal.pbio.3000385}
{PLoS Biology}



%%%%%%%%%%%%%%%%%%%%%%%%%%%%%%%%%%%%%%%%%%%%%%%%%%%%%%%%%%%%%%%%%%%%%%%%%%%%%%%

\SectionHead{\MakeUppercase{Awards, \mbox{Conferences}, \& Panels}}

Available upon request and online \href{https://github.com/thoppe/personal_cv/blob/master/tex/pdf/Travis_Hoppe_Federal_Resume.pdf?raw=true}{Federal Curriculum Vitae}.

%%%%%%%%%%%%%%%%%%%%%%%%%%%%%%%%%%%%%%%%%%%%%%%%%%%%%%%%%%%%%%%%%%%%%%%%%%%%%%%

\end{cleanCV}
\end{document}
