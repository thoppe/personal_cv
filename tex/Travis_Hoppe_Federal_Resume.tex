\documentclass[]{scrartcl}
\usepackage{src/cleancv}
\usepackage{enumitem}

\newcommand{\CVtype}{Federal Resume}
\newcommand{\CVname}{Travis Aaron Hoppe}
\newcommand{\CVemail}{travis.hoppe@gmail.com}
\newcommand{\CVdegrees}{PhD Physics}
\newcommand{\CVphone}{$(775) \ 287\hbox{-}4033$}

\titleformat{\section}[block]{\noindent\bfseries}{}{0em}{} 
\titleformat{\subsection}[block]{\noindent\itshape}{}{0em}{} 

\titlespacing*{\section}
{0pt}{2ex}{1ex}
\titlespacing*{\subsection}
{0pt}{1ex}{1ex}

\begin{document}
\begin{cleanCV}

   \SectionHead{\MakeUppercase{Qualifications}}
   \vspace{-1em}

   {
  \vspace{-0.25em}
  \begin{itemize}

      \item Respected and well cited author using AI/ML to address questions on multiple topics including: disparities in NIH funding (\href{https://www.science.org/doi/10.1126/sciadv.aaw7238}{Topic Choice}: 400 citations, key evidence used by the \href{https://bluntrochester.house.gov/uploadedfiles/191220_ltr_to_nih_about_grant_disparities.pdf}{Congressional Black Caucus}), open citations (\href{https://www.ncbi.nlm.nih.gov/pmc/articles/PMC6786512/}{NIH Open Citation Collection}: 70 citations), and a leading state-of-the-art AI model (\href{https://arxiv.org/abs/2101.00027}{The Pile}: 390 citations)

      \item Federal expert in data science and AI/ML. Led multiple teams across CDC to develop and implement cutting-edge policies, guidance, and strategies for AI/ML adoption, including conversational AI (Chat GPT), and infrastructure needs to support AI/ML implementation. Regular presenter across federal and industry venues on topics around Trustworthy AI, implementation, and development. Extensive experience in presenting to federal and industry audiences around Trustworthy AI, implementation, and development of new methodology.

  \end{itemize}
}

  \SectionHead{\MakeUppercase{Professional Experience}}

  \vspace{-1em}

\WorkExperience
{}
{Associate Director for Data Analytics and Data Science (ADDADS)}
{
  \newline Centers for Disease Control and Prevention (CDC), National Center for Health Statistics (NCHS)
  \newline December 2023 - Present
  \newline 40 hours per week
  \newline ST, 1560, \$176,043/year base compensation
  \newline
  \newline Senior Service Fellow
  \newline May 2022 - December 2023
  \newline 40 hours per week
  \newline RG, GS-15/2, 1530, \$160,899/year base compensation
}
{
  \vspace{-0.25em}
  \begin{itemize}

  \item Lead key aspects of a multi-year plan to modernize NCHS data systems through cloud adoption. Aligned and coordinated the plan with CDC's data modernization initiative (DMI) to help build the foundation for data sharing across all levels of public health. Coordinated policy, governance, and legal issues around the usage of CIPSEA protected data. Worked with Office of Management and Budget (OMB) to adopt CIPSEA guidance and lead NCHS as an early cloud adopter for statistical agencies.
    
        \item Piloted many innovative data science projects to help identity, analyze, and report on emerging public health data. Projects include an item non-response detection model for survey text which resulted in the \href{https://www.cdc.gov/nchs/data-science/SANDS-model-context.htm}{first AI model} released by NCHS, privacy enhancing technologies (PETs) like homomorphic encryption and privacy preserving record linkage, text-to-speech transcription which resulted in a 10-fold improvement from prior methods, and the creation of a bibliometrics dashboard for reporting.
    
  \item Developed the first set of model standards for AI/ML within the agency for trustworthy, responsible, and ethical usage. Aligned with existing standards including the NIST AI RMF and HHS Trustworthy AI.
    
  \item Developed, started, and led three different organizations within the agency: NCHS Innovation, NCHS Data Science Community of Practice, and the agency-wide EDAV Best Practices group. Organizations helped build community, foster communication, and develop innovative practices across CDC.

  \item Harmonized metadata usage for all NCHS datasets published on \href{https://data.cdc.gov/browse?category=NCHS}{data.cdc.gov}. Developed and implemented standards for tagging datasets around Social Determinants of Health (SDoH).
    
      \item Agency representative for National Science and Technology Council (NSTC) subcommittee on AI/ML and regular presenter at HHS AI and GSA AI communities of practice. Active committee member of the Federal Committee on Statistical Methodology (FCSM) co-leading a Federal level metadata initiative.

  \end{itemize}
}

\vspace{1em}

    
  \WorkExperience
{}
{Senior Service Fellow: Chief Data Scientist}
{
  \newline Centers for Disease Control and Prevention (CDC), National Center for
Health Statistics (NCHS)
  \newline September 2020 - May 2022
  \newline 40 hours per week
  \newline Title 42 (GS-14/5 equivalent) \$138,866/year base compensation with high-performance bonus 
}
{
  \vspace{-0.25em}
  \begin{itemize}

   \item Lead for NCHS Data Modernization Initiative: \emph{Increase Use, Discoverability, and Access to NCHS Data}. Established pilot projects and worked with stakeholders to create statements of work and business needs.
   \item Served on CDC's response to the Executive Order on AI and the NCHS Data Science Strategic Plan. Presented to Board of Scientific Council, NCHS All-Hands, and delivered subject matter talks on Bias in AI, advances in Natural Language Processing, bibliometrics, and more.
   \item Organized Center-wide metadata standards and built a custom ontology using evidence based sources: publications, web searches, and market research.
   \item Developed new methodology to study free text responses from the Research and Development Survey (RANDS), including non-response detection and zero-shot learning objectives.
   \item Implemented PII detection processes for restricted microlevel data.
  \end{itemize}
}
  
  \WorkExperience
{}
{Senior Data Scientist / Portfolio Analyst}
{
  \newline National Institutes of Health (NIH) / Division of Program Coordination, Planning, and Strategic Initiatives (DPCPSI) / Office of Portfolio Analysis (OPA) contracted under Lexical Intelligence
  \newline February 2016 - February 2020
  \newline 40 hours per week
  \newline \$135,000/year base compensation, yearly bonus of \$4,000
}
{
  \vspace{-0.25em}
  \begin{itemize}
    
    \item Scientific team leader for a novel inter-agency government blockchain to detect grant duplication with minimal shared data. Coordinated research, oversaw design, and developed protocols within the NIH and National Science Foundation (NSF) teams.

    \item Developed new analytic tools to process the text of NIH grants and publications using distributional embeddings (word2vec) and transformers (BERT). Tools were deployed for analysis presented to NIH senior leadership, Congress, and publications in high-ranking journals.

    \item Architected and productionized machine learning models for classification, regression, outlier detection, and language modeling. Creator and maintainer of several open-source tools used internationally in the scientific community.

      \item Trained and mentored junior staff in natural language processing (NLP) and machine learning.
  
  \item Analyzed grant and publication portfolios, evaluating metrics such as clinical impact, technological impact, and award rates to build quantitative comparisons between various populations.
    
  \item Restored historical texts from books and generated new structured data from free text. Expanded NIH grant coverage by thirty years from archival documents. Cross-linked publications to an NIH application's biographical sketch and literature cited. Data used internally with the NIH for analysis on racial disparity, topic analysis, mentorship, and grant efficacy.    
    
  \end{itemize}
}

\vspace{0.5em}
 
\WorkExperience
{}
{Postdoctoral Fellowship (IRTA) at National Institutes of Health}
{\newline Research Scientist}
{
  \newline April 2014 - February 2016
  \newline 40 hours per week
  \newline \$48,000/year
}
{
  \begin{itemize}
  \item Researched novel integration schemes for molecular dynamics simulations (MDS). Developed protein models for tertiary structure prediction from primary sequence.

  \item Designed and managed high-performance computing models on the NIH supercomputer, Biowulf. First to investigate containerized solutions for MDS using a graphics processing card.

  \item Worked in collaboration with experimentalists to test and validate models. 

  \end{itemize}
}

%\newpage

\WorkExperience
{}
{Postdoctoral Fellowship at National Institutes of Health}
{\newline Research Scientist}
{
  \newline August 2011 - April 2014
  \newline 40 hours per week
  \newline \$46,167/year
}
{
  \begin{itemize}
  \item Developed multi-scale theoretical and computational models to study protein folding, structure, and protein-protein. Derived hard-sphere models to account for crowding in biomolecular simulations and potentials to model anisotropic charge distributions.
  \item Managed large-scale parallel projects (1000+ cores) to simulate the cellular environment.
  \end{itemize}
}

\WorkExperience{}
{\newline Teaching Assistantship / Curriculum Designer}
{
  \newline September 2005 - May 2011
  \newline 35 hours per week
  \newline \$28,000/year
}   
{Teaching Assistant (Drexel)}

{
  \begin{itemize}
  \item   Organized, taught, and ran 22 undergraduate courses.
  \item   Personally restructured the entire computational component for physics majors by transitioning from FORTRAN to Python.
  \end{itemize}
}

%%%%%%%%%%%%%%%%%%%%%%%%%%%%%%%%%%%%%%%%%%%%%%%%%%%%%%%%%%%%%%%%%%%%%%%%%%%%%%%


\SectionHead{\MakeUppercase{Education}}

\WorkExperience
{2011}
{Doctor of Philosophy, Physics}
{
\\Drexel University
\\\emph{On the Role of Entropy in the Protein Folding Process}, \href{https://idea.library.drexel.edu/islandora/object/idea:3488}{Thesis}
}

\WorkExperience
{2008}
{Master of Science, Physics}
{\\Drexel University}

\WorkExperience
{2005}
{Bachelor of Science, Physics}
{\\University of Nevada}

\WorkExperience
{2005}
{Bachelor of Science, Mathematics}
{\\University of Nevada}
%%%%%%%%%%%%%%%%%%%%%%%%%%%%%%%%%%%%%%%%%%%%%%%%%%%%%%%%%%%%%%%%%%%%%%%%%%%%%%%

\SectionHead{\MakeUppercase{Committees served}}

\WorkExperienceX
{2021-Present}
{National Science and Technology Council (NSTC) subcommittee on AI/ML}
{CDC representative}

\WorkExperienceX
{2021-Present}
{Federal Committee on Statistical Methodology (FCSM)}
{Board Member}

\WorkExperienceX
{2022-Present}
{Building Trust and FAIRness into the Process  for Finding and Using Government Data (Chief Data Officers Council and FCSM)}
{Co-lead}

\WorkExperienceX
{2023}
{Utility and Risks to CDC of Conversational Artificial Intelligence (AI) Technologies like Chat GPT (CDC)}
{Tiger Team Lead}

\WorkExperienceX
{2023}
{Supporting the U.S. Public Health Workforce, (President's Council of Advisors on Science and Technology)}
{External Expert}



\WorkExperienceX
{2022}
{Analytics and Machine Learning Implementation within CDC's Cloud Environment (CDC)}
{Tiger Team Lead}

\WorkExperienceX
{2021}
{National Science and Technology Council (NSTC): Epidemic Modeling and Forecasting Fast Track Action Committee (FTAC)}
{Contributor: Plan to Advance Data Innovation}

\WorkExperienceX
{2021}
{Health and Human Services (HHS): Open Data Task Force}
{Committee member}

\WorkExperienceX
{2021}
{CDC: Information Technology and Data Governance (ITDG)}
{Committee member}


%%%%%%%%%%%%%%%%%%%%%%%%%%%%%%%%%%%%%%%%%%%%%%%%%%%%%%%%%%%%%%%%%%%%%%%%%%%%%%%
\newpage

\SectionHead{\MakeUppercase{Skills \& Clearance}}

\WorkExperience
{ }
{Machine learning / AI: }
{Natural Language Processing (NLP), Convolutional Neural Networks (CNN), Generative Adversarial Networks (GANs), Transformers (BERT, GPT), pyTorch, Tensorflow, word2vec}

\WorkExperience
{ }
{Programming \& Database: }
{Python, C\texttt{++}, JavaScript, SQL, NoSQL (MongoDB, Elasticsearch), Cloud Infrastructure (AWS, Azure, Heroku), Containerization (Docker, Kubernetes)}

\WorkExperience
{ }
{Project Management: }
{Experienced team leader for analysis, code, and design. Trained and mentored staff.}

\WorkExperience
{ }
{Public Trust Clearance: }
{Level 5, valid from 2018-2023}

%%%%%%%%%%%%%%%%%%%%%%%%%%%%%%%%%%%%%%%%%%%%%%%%%%%%%%%%%%%%%%%%%%%%%%%%%%%%%%%

%\newpage

\SectionHead{\MakeUppercase{Publications}}
\vspace{-1em}

\subsection{Policy}

\Paper
{2024}
{A Framework for Data Quality: Case Studies}
{Lisa Mirel, Darius Singpurwalla, Travis Hoppe, Rolf Schmitt, Julie Weber, Erika Liliedahl}
{https://doi.org/10.21949/1529869}
{Federal Committee on Statistical Methodology}
%{Data Quality Framework Implementation Subcommittee}

\Paper
{2023}
{Dark citations to Federal resources and their contribution public health}
{Jessica Keralis, Juan Albertorio-Díaz, \& Travis Hoppe}
{https://doi.org/10.3389/frma.2023.1235208}
{Frontiers in Research Metrics and Analytics}
%{https://www.biorxiv.org/content/10.1101/2023.03.26.533809v1}{bioRxiv}

\Paper
{2023}
{Application of a Novel Machine Learning Technique in a Bibliometric Analysis of Health Disparities Articles}
{Pascaline Ezouah, Bao-Ping Zhu, \& Travis Hoppe}
{}{Manuscript in preparation}

\Paper
{2019}
{Topic Choice Contributes to Lower Rate of NIH Awards to African-American/Black Scientists}
{Travis Hoppe, Aviva Litovitz, Kristine Willis, Rebecca Meseroll, Matthew Perkins, B. Ian Hutchins, Alison Davis, Michael Lauer, Hannah Valantine, James Anderson, \& George Santangelo}
{https://advances.sciencemag.org/content/5/10/eaaw7238}
{Science Advances}


\subsection{Data Science}


\Paper
{2024}
{Semi-Automated Nonresponse Detection for Open-text Survey Data}
{Kristen Cibelli Hibben, Zachary Smith, Ben Rogers, Valerie Ryan, Travis Hoppe}
{}
{Under review: Social Science Computer Review}


\Paper
{2023}
{Model Release: Semi-Automated Nonresponse Detection for Surveys (SANDS)}
{Kristen Cibelli Hibben, Zachary Smith, Ben Rogers, Valerie Ryan, Travis Hoppe}
{https://doi.org/10.57967/hf/0414}
{Hugging Face}

\Paper
{2023}
{Predicting causal citations without full text}
{Travis Hoppe, Salsabil Arabi, Ian Hutchins}
{https://www.pnas.org/doi/10.1073/pnas.2213697120}
{Proceedings of the National Academy of Sciences of the United States of America}

\Paper
{2023}
{Prediction of transformative breakthroughs in biomedical research}
{Matthew T. Davis, B. Ian Hutchins, Brad Busse, Payam Meyer, Grant Jones, Travis A. Hoppe, Kristine A. Willis, Abbey Zuehlke, Rebecca A. Meseroll, and George M. Santangelo}
{}{Manuscript in preparation}


\Paper
{2020}
{The Pile: An 800GB Dataset of Diverse Text for Language Modeling}
{Leo Gao, Stella Biderman, Travis Hoppe, \etal}
{https://arxiv.org/abs/2101.00027}
{arXiv}

\Paper
{2019}
{The NIH Open Citation Collection: A public access, broad coverage resource}
{Ian Hutchins, Kirk Baker, Matthew Davis, Mario Diwersy, Ehsanul Haque, Robert Harriman, Travis Hoppe, Stephen Leicht, Payam Meyer, George Santangelo}
{https://journals.plos.org/plosbiology/article?id=10.1371/journal.pbio.3000385}
{PLoS Biology}


\Paper
{2017}
{Additional support for RCR: A validated article-level measure of scientific influence}
{Ian Hutchins, Travis Hoppe, Rebecca Meseroll, James Anderson, \& George Santangelo}
{http://journals.plos.org/plosbiology/article?id=10.1371/journal.pbio.2003552}
{PLoS Biology}


\subsection{Protein Topology and Interactions}

\Paper
{2019}
{Non-specific Interactions Between Macromolecular Solutes in Concentrated Solution: Physico-Chemical Manifestations and Biochemical Consequences}
{Travis Hoppe \& Allen Minton}
{https://www.frontiersin.org/articles/10.3389/fmolb.2019.00010/full}
{Frontiers in Molecular Biosciences}


\Paper
{2016}
{Incorporation of Hard and Soft Protein-Protein Interactions into Models for Crowding Effects in Binary and Ternary Protein Mixtures}
{Travis Hoppe \& Allen Minton}
{http://pubs.acs.org/doi/abs/10.1021/acs.jpcb.6b07736}
{Journal of the Physical Chemistry B}

\Paper
{2015}
{Dependence of Internal Friction on Folding Mechanism}
{Wenwei Zheng, David De Sancho, Travis Hoppe \& Robert B. Best}
{http://pubs.acs.org/doi/abs/10.1021/ja511609u}
{Journal of the American Chemical Society}

\Paper
{2015}
{An equilibrium model for the combined effect of macromolecular crowding and surface adsorption on the formation of linear protein fibrils}
{Travis Hoppe, Allen Minton}
{http://www.sciencedirect.com/science/article/pii/S0006349514048115}
{Biophysical Journal}

\Paper
{2013}
{A simplified representation of anisotropic charge distributions in proteins}
{Travis Hoppe}
{http://link.aip.org/link/doi/10.1063/1.4803099}
{Journal of Chemical Physics}

\Paper
{2013}
{Singular Value Decomposition of the Radial Distribution Function 
for Hard Sphere and Square Well Potentials}
{Travis Hoppe}
{http://dx.doi.org/10.1371/journal.pone.0075792}
{PLoS ONE}

\Paper
{2010}
{Protein Folding with Implicit Crowders: 
  A Study of Conformational States Using the Wang-Landau Method}
{Travis Hoppe, Jian-Min Yuan}
{http://pubs.acs.org/doi/abs/10.1021/jp107809r}
{Journal of Physical Chemistry B}


\Paper
{2009}
{Entropic flows, crowding effects, and stability of asymmetric proteins}
{Travis Hoppe, Jian-Min Yuan}
{http://link.aps.org/doi/10.1103/PhysRevE.80.011404}
{Physical Review E}


\subsection{Graph theory}

\Paper
{2014}
{Integer sequence discovery from small graphs}
{Travis Hoppe, Anna Petrone}
{http://www.sciencedirect.com/science/article/pii/S0166218X15003704}
{Discrete Applied Mathematics}

\subsection{Experimental Modeling}

\Paper
{2014}
{Programmable Nanoscaffolds that Control Ligand Display to a G-Protein Coupled-Receptor in Membranes allow Dissection of Multivalent Effects}
{Andrew Dix, Daniel Appella, Travis Hoppe, \etal}
{https://pubs.acs.org/doi/abs/10.1021/ja504288s}
{Journal of the American Chemical Society}

\Paper
{2014}
{Quantification of plasma HIV RNA using chemically engineered peptide nucleic acids}
{Chao Zhao, Daniel Appella, Travis Hoppe, \etal}
{http://www.nature.com/ncomms/2014/141006/ncomms6079/abs/ncomms6079.html}
{Nature Communications}

\Paper
{2008}
{The importance of EBIT data for Z-pinch plasma diagnostics}
{A S Safronova, Travis Hoppe, \etal}
{http://dx.doi.org/10.1139/P07-170}
{Canadian Journal of Physics}

\Paper
{2006}
{Spectroscopic and Imaging Study of Combined W and Mo-pinches 
  at 1 MA-pinch Generators}
{Alla Safronova, Travis Hoppe, \etal}
{http://dx.doi.org/10.1109/TPS.2006.878361}
{IEEE Transactions on Plasma Science}



%%%%%%%%%%%%%%%%%%%%%%%%%%%%%%%%%%%%%%%%%%%%%%%%%%%%%%%%%%%%%%%%%%%%%%%%%%%%%%%

\SectionHead{\MakeUppercase{Conferences}}

\WorkExperienceX
{2023}
{Johns Hopkins Carey Business School: Conference on Health IT and Analytics (CHITA)}
{Panelist: AI Bias and Fairness Panel}

\WorkExperienceX
{2023}
{Health Datapalooza}
{Panelist: Federal Health, Responsible \& Trustworthy AI: From Principles to Practice}

\WorkExperienceX
{2023}
{Conference on Statistical Practice}
{Speaker: Examination of Dark Citations of Federal Information and their Contribution to Research}

\WorkExperienceX
{2022}
{General Services Administration AI Community of Practice}
{Invited Speaker: Detecting Non-response and PII in Web-based Surveys}

\WorkExperienceX
{2021}
{Federal Privacy Summit: Washington DC}
{Panel Chair: Automated PII detection}

\WorkExperienceX
{2021}
{Federal Committee on Statistical Methodology: Washington DC}
{Presentation: Short communication as a medium: Is Engagement a substitute for efficacy?}

\WorkExperienceX
{2021}
{Federal Committee on Statistical Methodology: Washington DC}
{Presentation coauthor: Analysis of Open-text Time Reference Web Probes on a COVID-19 Survey}

\WorkExperienceX
{2016}
{Biophysical Society: Los Angeles, CA}
{Poster: Coevolutionary signal enhancement}


\WorkExperienceX
{2015}
{Biophysical Society: Baltimore, MD}
{Seminar: Mean-field lattice-model IDPs, Binding Affinity \& Specificity}

\WorkExperienceX
{2014}
{Advances in Enhanced Sampling Algorithms: Telluride, CO}
{Seminar: Topological considerations in the Wang-Landau algorithm}

\WorkExperienceX
{2013}
{Biophysical Society: Philadelphia, PA}
{Seminar: Coarse-grained Electrostatic Models for Protein Solutions}

\WorkExperienceX
{2010}
{Biophysical Society: San Francisco, CA}
{Poster: Wang-Landau Density of States in Crowded Protein Environments}

\WorkExperienceX
{2009}
{Drexel University Libraries' Communication Symposium:\\The Hidden Costs of Scholarly Communication: Philadelphia, PA}
{Invited Panel Member}

\WorkExperienceX
{2009}
{Biophysical Society: Boston, MA}
{Poster: Exhaustive Properties of Simple Lattice Peptides}


%%%%%%%%%%%%%%%%%%%%%%%%%%%%%%%%%%%%%%%%%%%%%%%%%%%%%%%%%%%%%%%%%%%%%%%%%%%%%%%

\newpage

\SectionHead{\MakeUppercase{Awards}}

\WorkExperienceX
{2022}
{NCHS Director's Office Individual Merit Award}
{Data modernization and workforce coordination around data science}

\WorkExperienceX
{2022}
{NCHS Director's Office Group Merit Award}
{Release of two trillion accelerometer data points to the public from the Physical Monitoring Team}

\WorkExperienceX
{2017}
{Office of the Director's Honor Award}
{Outstanding support for the Grants Support Index \& Next Generation Research Initiative Analytical Team}

\WorkExperienceX
{2014}
{Top Presentation Award}
{Institution-wide recognition during the NIDDK Annual Conference.}

\WorkExperienceX
{2010}
{Research Assistant Grant}
{Competitive grant from Drexel Physics Department on the basis of outstanding research and teaching.}

\WorkExperienceX
{2010}
{Student Research Achievement Award (SRAA)}
{Top poster at the Biophysical Society 2010 meeting. }

\WorkExperienceX
{2009}
{Department Research Award (Senior Division)}
{Given by the Drexel Physics Department, this award recognized a high proficiency in both original research and synthesis of results into publications.}

\WorkExperienceX
{2008}
{Department Research Award (Junior Division)}
{Restricted to the first two years of study, the junior division award was awarded for early achievements in research.}

\WorkExperienceX
{2007}
{Teaching Assistant of the Year}
{Recognition by Drexel University as the top Teaching Assistant in the College of Arts and Sciences.}


%%%%%%%%%%%%%%%%%%%%%%%%%%%%%%%%%%%%%%%%%%%%%%%%%%%%%%%%%%%%%%%%%%%%%%%%%%%%%%%



%\newpage

\SectionHead{\MakeUppercase{Teaching Experience: \newline Drexel}}

\newcommand{\TeachingNote}{$^*$}

\Teaching
{2011}
{PHYS 305}{Computational Physics II\TeachingNote}

\Teaching
{2010}
{PHYS 304}{Computational Physics I\TeachingNote}
\Teaching
{}
{PHYS 160}{Introduction to Scientific Computing\TeachingNote}
\Teaching
{}
{PHYS 305}{Computational Physics II\TeachingNote}

\Teaching
{2009}
{PHYS 304}{Computational Physics I\TeachingNote}
\Teaching
{}
{PHYS 160}{Introduction to Scientific Computing\TeachingNote}
\Teaching
{}
{DSP 099}{Dragon Summer Program: Remedial Mathematics}
\Teaching
{}
{PHYS 100}{Preparation for Engineering Studies}
\Teaching
{}
{PHYS 305}{Computational Physics II\TeachingNote}

\Teaching
{2008}
{PHYS 304}{Computational Physics I\TeachingNote}
\Teaching
{}
{PHYS 102}{Fundamentals of Physics II\TeachingNote}
\Teaching
{}
{PHYS 115}{Contemporary Physics III\TeachingNote}
\Teaching
{}
{PHYS 114}{Contemporary Physics II\TeachingNote}

\Teaching
{2007}
{PHYS 113}{Contemporary Physics I\TeachingNote}
\Teaching
{}
{PHYS 102}{Fundamentals of Physics II, Lab}
\Teaching
{}
{PHYS 115}{Contemporary Physics III\TeachingNote}
\Teaching
{}
{PHYS 114}{Contemporary Physics II\TeachingNote}

\Teaching
{2006}
{PHYS 113}{Contemporary Physics I\TeachingNote}
\Teaching
{}
{TDEC 101}{Fundamentals of Physics I, Lab}
\Teaching
{}
{TDEC 103}{Fundamentals of Physics III\TeachingNote}
\Teaching
{}
{TDEC 102}{Fundamentals of Physics II\TeachingNote}

\Teaching
{2005}
{TDEC 101}{Fundamentals of Physics I\TeachingNote}

\TeachingNote{\small Developed new curricula and modernized the Computational Physics, Contemporary Physics and Introduction to Scientific Computing courses at Drexel.}

\end{cleanCV}
\end{document}
